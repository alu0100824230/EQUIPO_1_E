%%%%%%%%%%%%%%%%%%%%%%%%%%%%%%%%%%%%%%%%%%%%%%%%%%%%%%%%%%%%%%%%%%%%%%%%%%%%%
% Chapter 4: Conclusiones y Trabajos Futuros 
%%%%%%%%%%%%%%%%%%%%%%%%%%%%%%%%%%%%%%%%%%%%%%%%%%%%%%%%%%%%%%%%%%%%%%%%%%%%%%%
Tras haber realizado una serie de experimentos con el programa implementado, podemos afirmar que \'este es eficaz para la funci\'on trabajada. \'Esto se consigue al utilizar distintos intervalos, repetir los mismos, y elegir diferentes funciones y es que el programa se encargar\'a de mostrarnos la existencia, o no, de una ra\'iz real. Es decir, si podemos hallar un punto de dicha funci\'on que corte al eje de abscisas.
\medskip


As\'i, hemos conseguido un programa que nos trabaja con cualquier tipo de funci\'on y hemos probado, experimentalmente, c\'omo se aproximan
las ra\'ices y c\'omo funciona el m\'etodo de bisecci\'on.
\medskip


Sin embargo, para la funci\'on escogida, quiz\'as podemos considerar como m\'etodo m\'as r\'apido, resolver la ecuaci\'on obtenida al igualar
a cero  nuestra funci\'on. Adem\'as de tener que considerar el hecho de que, si trabajamos con una funci\'on con m\'as de una ra\'iz, y elegimos
un intervalo en el que se encuentran, tambi\'en, m\'as de una, nuestro programa dar\'a error, puesto que solo funcionar\'a para intervalos en los que
encontremos un \'unico punto de corte.
\medskip


En definitiva, los experimentos llevados a cabo nos muestran una buena ejecuci\'on del programa para funciones similares a la escogida, o para polinomios
de grado mayor o igual a dos. En este \'ultimo caso, tendremos que conocerlos intervalos en los que podamos encontrar cada una de las ra\'ices y recordar que han de ser intervalos diferentes, para que as\'i nuestro programa no cause problema alguno, pudiendo ser ejecutado.
