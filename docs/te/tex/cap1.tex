%%%%%%%%%%%%%%%%%%%%%%%%%%%%%%%%%%%%%%%%%%%%%%%%%%%%%%%%%%%%%%%%%%%%%%%%%%%%%
% Chapter 1: Motivaci�n y Objetivos 
%%%%%%%%%%%%%%%%%%%%%%%%%%%%%%%%%%%%%%%%%%%%%%%%%%%%%%%%%%%%%%%%%%%%%%%%%%%%%%%
En la gu�a docente de la asignatura de T�cnicas Experimentales de Primero de Grado en Matem�ticas surge como una de las competencias b�sicas la realizaci�n de un experimento del que posteriormente se realizar� un informe y una presentaci�n. Para ello, los alumnos deben emplear las herramientas inform�ticas explicadas durante las clases.

De este modo, el alumno debe analizar, sintetizar, evaluar y describir los datos obtenidos del estudio de la funci�n propuesta. Escoger un intervalo espec�fico, realizar las operaciones correspondientes al m�todo de bisecci�n y obtener un resultado que ser�, posteriormente, evaluado y analizado cuantitativamente de forma experimental. Representar gr�ficamente los resultados obtenidos, sintetiz�ndolos y exponi�ndolos de forma objetiva. Utilizar herramientas inform�ticas, programando en un lenguaje relevante para el c�lculo cient�fico (python, \LaTeX{}, beamer).
\begin{itemize}
  \item {\bf Objetivo principal}: Implementaci�n con Python del m�todo de bisecci�n.
  \item {\bf Objetivo espec�fico}: C�mo se aproximan las ra�ces de una funci�n, mediante el m�todo de bisecci�n.
\end{itemize}
%---------------------------------------------------------------------------------
\section{Objetivo Principal}
\label{1:sec:1}
El objetivo principal es implementar el algoritmo del m�todo de bisecci�n en un intervalo [a,b], tal que $f(a)*f(b)<0$:\medskip
\begin{enumerate}
\item Se toma $c=\frac{(b-a)}{2}$
\item Si $b-a\le error$ se acepta c como la ra�z y se para.
\item Si $f(b)*f(c)\le0$, se toma $a=c$, por el contrario hacer $b=c$.
\end{enumerate}
%---------------------------------------------------------------------------------
\section{Objetivo Espec�fico}
\label{1:sec:2}
Mediante el m�todo citado obtenemos la ra�z �nica ($x=0$) de la funci�n:
\[f(x)=5^{x}-5\]